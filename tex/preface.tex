\section*{Preface}
\addcontentsline{toc}{section}{Preface}

\textbf{NOTE: Some sections are incomplete, particularly the examples section. I am updating now, so as the current title suggests, came back in a weeks time and I will have it all tidied up \smiley}
%
%
\par
%
%
As the size of data sets increase and models become increasingly complex, the need for High Performance Computing (HPC) systems becomes increasingly prominent. The purpose of this document is to provide a gentle introduction to using the HPC cluster from a statistical perspective. Sample scripts and examples are provided, and intended to serve as a basis for any future work people may need. You will be able to modify the example provided in this guide to your needs to make submitting jobs to HPC as easy as possible. It is hoped that this document can serve as a reference guide, a first port of call to those who might not be familiar with HPC and want to benefit from the facilities. It is also important to note that the scheduler used at QUT is used extensively at many universities and research facilities, so knowledge of QUT's HPC system will extend much past QUT.
%
%
\par
%
%
This guide will start with a brief introduction to UNIX like systems, and how to navigate a command line. The guide will also show how you can bypass much of the command line use, to remove as much prerequisite knowledge of UNIX as possible. This is achieved by mounting your HPC home drive onto your local machine, so that you can edit source code using editors that you are already familiar with. The HPC scheduler is then introduced, along with modules currently installed on the cluster and how to access them. Sample scripts are provided with instructions on how to submit both batch and interactive jobs. Instructions on how to customise your personal environment and install your own packages is also provided. This guide will primarily focus on programs written in R, though instructions can be followed with only slight modifications for other platforms and languages such as MATLAB, Python and Mathematica.
%
%
\par
%
%
Additional tips and tricks to help get the most out of the cluster are also provided. The tips include suggestions on how to (hopefully) simplify your HPC experience. This will be demonstrated through sample scripts that you can run yourself.
%
%
\par
%
%
From my experience, others have often already encountered and solved many of the problems that I have faced. I have also noticed that more often than not, others with more experience than myself have developed far more elegant solutions to these same problems. It is for this reason that I have decided to make this document, to impart some of the knowledge that others with considerably more experience have passed on to me. In this spirit, I am happy for this to be a living document that all can edit, to provide your own tips and tricks to solve potential problems you think others might encounter. Through collaboration, we can all benefit from each others work and streamline our development process.
%
%
%
%
%%% Local Variables:
%%% mode: latex
%%% TeX-master: "main"
%%% End:

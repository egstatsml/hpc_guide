\section{Submitting Jobs}
\label{ref:submitting}
%
%
%
Whilst it may seem like ample resources are available, they are finite and accessed by many people, thus access to these services needs to be managed. Access to computing resources is scheduled by the Portable Batch System (PBS). Before you run any software, you must tell the PBS manager what resources you require and how long you will need them for. This is done by submitting jobs.
%
%
\par
%
%
There are two main types of jobs you can submit to HPC; interactive and batch jobs. Interactive jobs provide you with an active terminal similar to what you will currently use on your desktop. Interactive jobs are useful for debugging and ensuring all of your code can run.
%
%
\par
%
%
Interactive jobs are often convenient, though you need to have your interactive session open for the job to continue. Batch jobs are intended for jobs that will need to run for longer, or when multiple jobs need to be submitted. Unlike interactive jobs, after you submit a batch job, you can close your connection to HPC altogether, go get a coffee, put your feet up and relax while your job processes on HPC in the background.
%
%
%
\par
We will first start with interactive jobs, and then show how we can move our work to batch jobs later on.
%
%
%
\subsection{Interactive Jobs}
To submit a job to HPC, we use the \hltexttt{qsub} command, along with a few arguments to tell the scheduler what resources we require, and how long we need them for. To submit an interactive job, run the following command,
%
%
\\
\par
\begin{minted}[bgcolor=Light, frame=single, fontsize=\footnotesize, fontfamily=courier]{bash}
  #submit an interactive job to HPC
  #replace the HH:MM:SS with the amount of time you
  #expect your job to run
  #Replace XXX with the amount of RAM you need
  #Replace YYY with the number of CPUS you need
  #Replace ZZZ with the name of your job
  #you can call it whatever you like :)
  qsub -I -S /bin/bash -l walltime=HH:MM:SS,mem=XXXg,ncpus=YYY -N ZZZ
\end{minted}
%
\\
%
Change the variables supplied here with the time and resources required for your work. After running this command, you will have to wait for your requested resources to be allocated to you. The time taken for your job to be accepted will depend on the amount of resources you requested. If you asked for 8 GB of RAM, 2 CPUs for only a couple hours, your job should be accepted within a minute or so. If you request 100 GB, 20 CPUs for 12 days, expect to wait a very long time for your interactive job to be accepted.\footnote{You can request these resources, but will need to submit a batch job. See section \ref{sec:batch} for instructions on how to submit these.}

\subsection{Modules}
Now that you have transferred your code across over to HPC and have been allocated resources for a job, you can start loading and installing the required packages you need. HPC has many programs already installed, though they aren't initially loaded when you log in. These pre-installed programs are stored as \textit{modules} that need to be first loaded before you can use them. To see the modules currently installed on HPC, run the command,
%
%
\\
\par
\begin{minted}[bgcolor=Light, frame=single, fontsize=\footnotesize, fontfamily=courier]{bash}
  #see what modules are available
  module avail
\end{minted}
From the output of this, you may begin to appreciate why not all of the packages are loaded on startup, there is an awful lot of them. You can search through the output to find any modules you are interested in. Once you have found the module you are interested in, you can load it with the \hltexttt{module load} command. An example of common modules that might be helpful are listed here.
%
\\
%
\subsubsection{R Modules}
Output of \hltexttt{module avail r}
\begin{minted}[bgcolor=Light, frame=single, fontsize=\footnotesize, fontfamily=courier]{bash}
  #load in R (output of module avail r)
  -------------- /pkg/suse12/modules/lang --------------
  r/3.1.0-foss-2016a              r/3.4.2-foss-2017a             
  r/3.2.0-foss-2015a-bare         r/3.5.1-foss-2018a             
  r/3.3.1-foss-2016a              r/3.6.1-foss-2019a             
  r/3.3.1-intel-2016b(default)    r/3.6.2-foss-2019b
  r/3.4.2-bioconductor-foss-2017a r/4.0.3-foss-2020b
\end{minted}
%
%
\subsubsection{Matlab Modules}
Output of \hltexttt{module avail matlab}
\begin{minted}[bgcolor=Light, frame=single, fontsize=\footnotesize, fontfamily=courier]{bash}
  -------------- /pkg/suse12/modules/math --------------
  matlab/2016b matlab/2017a matlab/2017b
  matlab/2018b matlab/2019a matlab/2021a
\end{minted}
%
%
\subsubsection{Mathematica Modules}
Output of \hltexttt{module avail mathematica}
  \begin{minted}[bgcolor=Light, frame=single, fontsize=\footnotesize, fontfamily=courier]{bash}
  -------------- /pkg/suse12/modules/math --------------
  module load mathematica/11.2.0-linux-x86_64
\end{minted}

\subsubsection{Python Modules}
Output of \hltexttt{module avail python}
\begin{minted}[bgcolor=Light, frame=single, fontsize=\footnotesize, fontfamily=courier]{bash}
  -------------- /pkg/suse12/modules/lang --------------
python/2.7.10-intel-2015b                  python/2.7.15-intel-2018b
python/2.7.11-foss-2016a                   python/2.7.16-foss-2019a
python/2.7.11-intel-2016a                  python/2.7.16-gcccore-7.3.0
python/2.7.11-intel-2016b                  python/2.7.16-gcccore-8.3.0
python/2.7.11-iomkl-2016.07                python/2.7.18-gcccore-10.2.0
python/2.7.11-iomkl-2016.09-gcc-4.9.3-2.25 python/2.7.18-gcccore-9.3.0
python/2.7.12-foss-2016a-repair            python/2.7.5-foss-2017a
python/2.7.12-foss-2016b                   python/3.5.1-foss-2016a(default)
python/2.7.12-gmpolf-2016b                 python/3.5.2-foss-2016b
python/2.7.12-intel-2016a                  python/3.5.2-intel-2016b
python/2.7.12-intel-2016b                  python/3.6.1-foss-2016a
python/2.7.13-foss-2017a-foss              python/3.6.1-foss-2017a
python/2.7.13-foss-2018a                   python/3.6.1-foss-2017a-busco
python/2.7.13-intel-2017a                  python/3.6.1-intel-2017a
python/2.7.13-iomkl-2016.07                python/3.6.2-foss-2017b
python/2.7.14-foss-2016a                   python/3.6.3-foss-2017b
python/2.7.14-foss-2016a-tf                python/3.6.4-foss-2016a
python/2.7.14-foss-2017b                   python/3.6.4-foss-2018a
python/2.7.14-foss-2018a                   python/3.6.4-foss-2018a-pytorch-0.3.1
python/2.7.14-gcccore-6.4.0-bare           python/3.6.4-foss-2018a-pytorch-0.5.0
python/2.7.14-intel-2017a                  python/3.6.4-intel-2017a
python/2.7.15-foss-2016a                   python/3.7.1-foss-2018a
python/2.7.15-foss-2018a                   python/3.7.2-foss-2018a
python/2.7.15-foss-2018b                   python/3.7.2-gcccore-8.2.0
python/2.7.15-foss-2019a                   python/3.7.2-gcccore-8.3.0
python/2.7.15-foss-2019b                   python/3.7.4-gcccore-8.3.0
python/2.7.15-fosscuda-2018b               python/3.8.2-gcccore-9.3.0
python/2.7.15-gcccore-7.3.0-bare           python/3.8.6-gcccore-10.2.0
python/2.7.15-gcccore-8.2.0                python/3.9.1-gcccore-10.2.0
python/2.7.15-gcccore-8.3.0                python/3.9.1-gcccore-9.3.0
python/2.7.15-gcccore-8.3.0-bare
\end{minted}
%
\\
\par
%
\begin{story}
  \textbf{NOTE:}
  \\
  Python2 is now deprecated. Don't use it for new projects.
\end{story}
%
%
For this guide, we will use R and Python as examples, though you can adapt it for other programming languages with only small modification. So first we load in the R module with \hltexttt{module load load r/3.6.2-foss-2019b}. Once loaded, you can start R by simply typing \hltexttt{R} into the terminal.
%
%
%
\par
You can list all modules that you have loaded in using the following command,
\\
\par
\begin{minted}[bgcolor=Light, frame=single, fontsize=\footnotesize, fontfamily=courier]{bash}
  #list all modules we have loaded in to our current environment
  module list
\end{minted}

If you have found that you have loaded in an incorrect module, or you have loaded in multiple modules for the same program (ie loading in multiple versions of R/MATLAB etc.) you can start from a clean slate by purging all loaded modules by running the command,
%
\\
\par
\begin{minted}[bgcolor=Light, frame=single, fontsize=\footnotesize, fontfamily=courier]{bash}
  #unload all modules you have loaded in
  module purge
\end{minted}
%
%
%
\subsection{Installing Additional Packages}
%
%
For your given projects, you are more than likely going to need to install some additional packages, regardless of your programming language of choice.
Im going to go through a quick rundown of how I would suggest doing this for both R and Python, as I believe these show the two different methods to seprate packages from different language versions; one being relatively easy, and the other being more difficult (at least in my opinion).
%
%
\subsubsection{Installing Python Packages}
\label{sec:install_python}
% 
%
Installing additional Python packages is relatively simple, and for the most part pretty quick. The main component that makes it a bit easier than other languages is the virtual environment \hltexttt{virtualenv} features. A virtual environment in Python allows you to have an isolated, environment wide container for a specific project. This makes it really easy to install new packages for specific project, a specific compiler settings or Python version. The basic procedure is,
%
%
%
\begin{enumerate}
  \item Load in the Python Module we want
  \item Create virtual environment
  \item Load virtual environment
  \item Install packages as normal
\end{enumerate}
%
%
Once you have loaded the virtual environment you have created, installing packed like you normally would with \hltexttt{pip} will just go straight under this isolated environment. An example could look like,
%
%
%
\begin{minted}[bgcolor=Light, frame=single, fontsize=\footnotesize, fontfamily=courier]{bash}
  # submit an interactive job to HPC
  # Reminder, always submit a job before you do pretty much anything
  qsub -I -S /bin/bash -l walltime=HH:MM:SS,mem=XXXg,ncpus=YYY -N ZZZ
  # lets just make sure we are starting from our home directory
  cd ~
  # load in the python module (1)
  module load python/3.7.4-gcccore-8.3.0
  # create the virtualenv (2)
  # im going to call this virtual environment p_3.7.4
  python -m venv p_3.7.4
  # activate the virtual environment (3)
  source ~/p_3.7.4/bin/activate
  # install packages like we normally would
  pip install <package_name>
  # if you need to deactivate the virtual environment, can just run
  deactivate
\end{minted}
%
%
And that is pretty much it for Python. Most of the packages should install pretty quickly, so I would recommend submitting an interactive job like this and then just manually installing all the libraries you need.
%
%
\par
%
%
It is natural to think, why am I bothering with virtual environments here? In the short term, you could very well get away without using them, but in the long run a new version and module for python will come out and you will want to switch to that. Having a virtual environment for a specific Python version allows you to move on to a new version of Python without getting weird conflict issues or compilation errors. Instead of trying to manage the conflicts within the changfe of Python versions, I can just simply create a new virtual environment and install everything I need again. It might seem like installing everything again will take a long time, but the reality is it will take almost no time when compared with the alternative of managing conflicting package versions, compiler versions etc.
%
%
%
\par
%
%
%
Using virtual environments is also just general good practice. Will often find that certain packages require exact, explicit versions of another package. It is often the case that installing a specific version of a package can break a dependency of another more important package you already have installed. With a virtual environment, you can solve this issue without too much of an issue. You can create a new virtual environment for the problematic package with the strict and incompatible dependency, and just activate that whenever you need it. 
%
%
%
\subsubsection{Installing R Packages}
\label{sec:install}
Now for R's turn at installing packages, and it is a little bit tougher for R (but only a little bit). The main reason it is tougher is because virtual environments aren't as well supported in R to help keep things isolated\footnote{Though it looks like there is active development to make it a bit nicer https://rstudio.github.io/renv/articles/renv.html}.
%
%
\par
%
%
%
Similar to Python above, we want to install new packages that are compatible with the R version we are using. For example, at the moment you may be using R 3.6.2, but maybe the time will come where you are forced to move to 4.0.3? When making a shift, it is expected to encounter conflicts with existing versions, particularly with R since many packages rely on compiled \hltexttt{Rcpp} code. For this reason, I am making the install procedure slightly more difficult to enable us to have a more general installation method that will allow us to move to a new version of R when needed\footnote{If anyone has a better way to make the installation of new packages flexible and simpler, please let me know and I would be happy to update it. I tried not explicitly setting the \hltexttt{lib} directory for different R versions, but R still installed user packages to the same location, and this is what I want to avoid.}.
%
%
%
\par
%
%
%
To show how to install packages for R, I have provided an example script \hltexttt{install\_r\_packages.R}, which will install many of the common packages you will need (some of them are likely already installed though). After you have cloned the example repository listed earlier, you can run the script to install all of the base packages with these commands for R 3.6.2,
%
%
%
\par
\begin{minted}[bgcolor=Light, frame=single, fontsize=\footnotesize, fontfamily=courier]{bash}
  # submit an interactive job to HPC
  # Reminder, always submit a job before you do pretty much anything
  qsub -I -S /bin/bash -l walltime=HH:MM:SS,mem=XXXg,ncpus=YYY -N ZZZ
  # load in the R module
  module load r/3.6.2-foss-2019b
  # create directory where user packages will be installed
  mkdir -p R/library_3.6.2
  # set the R environment variable to say this is where to look
  # for additional user libraries
  echo 'R_LIBS_USER="~/R/library_3.6.2"' > ~/.Renviron
  # lets just make sure we are starting from our home directory
  cd ~
  # pull the repo for this guide
  git clone https://github.com/ethangoan/hpc_guide
  # change directory to the repo
  cd hpc_guide/install_r_packages
  # now run the install script using the install_r_packages.sh script
  # the additional command line argument here tells the script where to
  # install the new packages to.
  Rscript install_r_packages.R ~/R/library_3.6.2
\end{minted}
%
%
%
\par
%
%
%
This will take a while to run (a few hours I think), so you can either leave your terminal open and let the program run, or you can use the instructions in the next section, where we will learn to submit a batch job that will install all of the packages for you.
%
\begin{story}
  \textbf{NOTE:}
  \\
  If you ran the above interactive script to install all the packages, once it is done there is one more command you will need to run. In this script, packages will be installed into \texttt{\textasciitilde R/library} directory. This needs to be done because installing packages on HPC is slightly different to that of your desktop machine, as you don't have root access on HPC. To rectify this, we just need to tell R where to look to find the installed packages. To do this, run the command
  \\
  \begin{minted}[fontsize=\footnotesize, fontfamily=courier,escapeinside=||]{bash}
    echo 'R_LIBS_USER="~/R/library"' > |~/.Renviron|
  \end{minted}
  in the terminal. If you install all the packages using the batch script example in the next section, you won't need to run this command, it will do it for you.
\end{story}
%
%
%
%
\subsection{Submitting Batch Jobs}
\label{sec:batch}
In the previous example, we saw how to submit an interactive job, load in R or Python modules and install some base packages in an interactive session. We can achieve this same result by submitting a batch job, which will run on HPC without us having to intervene and leave the terminal open. Batch jobs are useful for programs that require a long time to run, since we can simply submit them and then forget about them (while they running at least).
%
%
\par
%
%
Like submitting an interactive job, we need to specify the time and computational resources we require. Unlike interactive jobs, we specify these requirements through a configuration file. In the guide repository, an example batch configurations script called \hltexttt{install\_r\_packages/install\_packages\_batch.sh} and \hltexttt{install\_python\_packages/install\_packages\_batch.sh} are supplied. These are Bash scripts that is interpreted by the PBS scheduler, and specifies our requirements and which program we want to run. Computational requirements are listed at the top of the file in the commented out section. These are called the PBS directives, and are shown as an example below.
%
%
\\
%
%
\par
\begin{minted}[bgcolor=Light, frame=single, fontsize=\footnotesize, fontfamily=courier]{bash}
  #!/usr/bin/env bash
  
  #PBS -N install_packages
  #PBS -l ncpus=1
  #PBS -l mem=2GB
  #PBS -l walltime=20:00:00
  #PBS -o install_packages_stdout.out
  #PBS -e install_packages_stderr.out
  
\end{minted}
%
\\
%
The main differences here is the first line which is called the Shebang. This MUST be there in any batch configuration script, you will never need to change it. The other differences is the last two lines, which specifies where standard output and error messages will be written to.
%
%
\par
%
%
Further down in the script you will see helper functions that will load all of the modules we need (for this example we only need the R or python module) and a function which invokes the R script to install the packages we need (for Python there isn't a python script called, as we can do it easily from the bash whilst using virtual environments). These helper functions are called at the end of the script when the job has been submitted and accepted by the scheduler. We can submit this job using the following command,
%
%
\\
\par
\begin{minted}[bgcolor=Light, frame=single, fontsize=\footnotesize, fontfamily=courier]{bash}
  #clone the guide repo into your home directory
  #if you haven't already
  cd ~
  git clone https://github.com/ethangoan/hpc_guide
  #change into repo directory
  cd ~/hpc_guide
  # PYTHON PACKAGES INSTALL EXAMPLE
  # submit the batch job with qsub to install
  # python packages under the virtual environment specified earilier
  # change into required directory then subit batch job with qsub
  # NOTE: this assumes you already made the virtual environment
  # discussed in earlier sections
  cd ./install_python_packeges
  qsub install_packages_batch.sh
  # R PACKAGES INSTALL EXAMPLE
  # move back to the root directory for this guide
  cd ~/hpc_guide
  # change to R directory to install packages with script
  cd ./install_r_packeges
  qsub install_packages_batch.sh
\end{minted}
\\
%
Once you have submitted the job, you can track all of your submitted jobs using the command,
%
\\
\par
\begin{minted}[bgcolor=Light, frame=single, fontsize=\footnotesize, fontfamily=courier]{bash}
  watch -n 1 qstat -u <your_QUT_username>
\end{minted}
%
\\
\\
%
This will give you information on all the jobs you have submitted. You will be able to see whether they have commenced running, or if they are still running and how long they have been running for. Once the program has finished running, you can view the output of the installation script with the \hltexttt{cat} command.
\\
\par
\begin{minted}[bgcolor=Light, frame=single, fontsize=\footnotesize, fontfamily=courier]{bash}
  #check the output of the program
  cat install_packages_stdout.out
  #check the error log to see if anything went wrong
  cat install_packages_stderr.out
\end{minted}
%
%
%
\subsubsection{Installing More Packages}
%
%
While running these installation script will install some common packages, it is unlikely that it will install everything you require. To install more packages for R, I would suggest modifying the \hltexttt{install\_r\_packages.R} script to include packages to want to install. There is a slight difference to installing packages when compared with a typical desktop machine you own. Since you won't have root/administrator access on HPC, you will need to install the packages locally. The \hltexttt{install\_r\_packages.R} installs the packages locally and sets the relevant path variables so that R can find the packages we installed. To install more packages, simply edit the \hltexttt{packages} vector in that script and resubmit the batch job using the same commands as before.
%
%
\par
%
%
%
For Python, I would recommend using the interactive method used earlier in Section \ref{sec:install_python}.
%
%
\subsubsection{Quick Note on Python}
%
%
While there aren't many pre-installed packages for R on HPC, there is many for Python. Popular packages such as Numpy, Matplotlib, Scipy, Sklearn etc. are already installed and have their own module. To find these modules, simply run the \hltexttt{module avail} command and search for the module you require. Then load the module using the same \hltexttt{module load} command used previously.
%
%
%
%%% Local Variables:
%%% mode: latex
%%% TeX-master: "main"
%%% End:
